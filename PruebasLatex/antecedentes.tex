\chapter{Antecedentes}
\label{antecedentes}

\section{Dispositivos m{\'{o}}viles}

Un dispositivo m{\'{o}}vil es un dispositivo electr{\'{o}}nico 
\citet{ChallengeMobile} que posee una interfaz m{\'{a}}s pequeña 
comparada a la de otros dispositivos. Ejemplos de dispositivos 
m{\'{o}}viles son la computadora de a mano (pie de nota PDA - 
asistente digital personal) y los tel{\'{e}}fonos m{\'{o}}vil. 
Conforme los tel{\'{e}}fonos m{\'{o}}viles evolucionaron 
\citet{MobileEducation}, estos dispositivos comenzaron a ser m{\'{a}}s amigables para el usuario y se volvieron m{\'{a}}s "inteligentes" por sus nuevas funcionalidades, con lo que se les adjudic{\'{o}} el nombre de "smartphones" (pie de nota?).

Existen muchas definiciones para los smartphone, pero por lo general se acepta \citet{MobileEducation} que son tel{\'{e}}fonos m{\'{o}}viles que cumplen con sus funciones b{\'{a}}sicas y que adem{\'{a}}s poseen caracter{\'{i}}sticas propias de una computadora de escritorio, como un sistema operativo y la posibilidad de agregarle nuevo software.

\subsection{Android/iOS}

¿?

\section{Desarrollo de software}

¿? Ciclo de software...

\subsection{Desarrollo de software en dispositivos m{\'{o}}viles}

¿?

\section{Teor{\'{i}}a de la comunicaci{\'{o}}n}

Un mensaje puede ser \citet{Shannon_Theory} una secuencia de letras, una funci{\'{o}}n en el tiempo, etc{\'{e}}tera. Para que algo sea considerado mensaje, {\'{e}}ste debe tener un significado y ser relevante para alguien o algo. Los mensajes poseen un papel importante en la comunicaci{\'{o}}n.

Un sistema de comunicaci{\'{o}}n consta [Shannon] de cinco elementos:

Una fuente de informaci{\'{o}}n. Es qui{\'{e}}n produce el mensaje o secuencia de mensajes que ser{\'{a}}n comunicados a su destinatario. Tambi{\'{e}}n se le conoce como emisor.
Un transmisor. Es un componente que transforma el mensaje en una señal que sea apropiada para el canal de comunicaci{\'{o}}n utilizado.
Un canal de comunicaci{\'{o}}n. Es el medio por donde se transmite la señaldel emisor al destinatario.
Un receptor. Es un componente que realiza la operaci{\'{o}}n inversa del transmisor, es decir, reconstruye el mensaje original de la señal.
Un destinatario. Es la persona (o cosa) qui{\'{e}}n recibe el mensaje.

\section{Arquitectura de computadoras}
\subsection{Archivos digitales}

¿? Bytes, bits, conjunto de instrucciones, archivos binarios, archivos de texto, archivos multimedia,

The Essentials of Computer Organization and Architecture (Linda Null) P{\'{a}}gina 284, 285

\subsection{Im{\'{a}}genes digitales}

¿? Resoluci{\'{o}}n, formato, pixeles.


\section{Seguridad inform{\'{a}}tica}

La seguridad inform{\'{a}}tica, tambi{\'{e}}n conocida como seguridad 
digital, hace referencia \citet{Secrets_Schneier} a toda una serie de elementos que permite lograr "seguridad" en sistemas digitales, de acuerdo a ciertos contextos.

Para Bruce Schneier, \citet{Secrets_Schneier}, la seguridad es una cadena que 
involucra muchos componentes, es decir, la seguridad no es un producto que se pueda ofrecer, sino un proceso que cambia con el paso de tiempo y que s{\'{o}}lo se puede garantizar con un correcto diseño de toda infraestructura tecnol{\'{o}}gica y humana.

\subsection{Seguridad por obscuridad}

La idea de "seguridad por obscuridad" significa que \citet{Obscurity} la seguridad de un sistema se respalda en la creencia de qu{\'{e}} el atacante ignora las cualidades del sistema. Por ejemplo, una compañ{\'{i}}a puede ocultar que existe un servidor y que s{\'{o}}lo es conocido por pocas personas. No obstante, este enfoque, al menos en el caso descrito, no es muy confiable, pues con distintas herramientas es posible descubrir este servidor, rompiendo cualquier seguridad que tuviera.

La seguridad por obscuridad es desconsejado por muchos 
acad{\'{e}}micos \citet{Obscurity}, en especial en el campo de la 
criptograf{\'{i}}a, no obstante, su concepto es importante en otros 
campos, como por ejemplo, en la esteganograf{\'{i}}a 
\citet{Obscurity_Stegano}.

\subsection{Criptograf{\'{i}}a}

Hola

\subsection{Esteganograf{\'{i}}a}

La esteganograf{\'{i}}a es un t{\'{e}}rmino que hace referencia a la 
actividad de ocultar mensajes secretos dentro de otros mensajes 
\citet{AC_Schneier} , de tal manera que el primer mensaje sea oculto. Para lograr esto, el emisor manipula un mensaje inofensivo, ya sea creado por {\'{e}}l o que ya existe, y lo manipula de tal manera que logra su objetivo de ocultar el mensaje deseado sin que el otro mensaje posea cambios notorios .

El uso de la esteganograf{\'{i}}a se remota a tiempos remotos 
\citet{AC_Schneier}, en la que algunas de las t{\'{e}}cnicas con la que se realizaba inclu{\'{i}}a el uso de tinta invisible o remarcar letras en un texto escrito.

Hoy en d{\'{i}}a, la esteganograf{\'{i}}a es m{\'{a}}s 
com{\'{u}}nmente asociada con gr{\'{a}}ficos por computadora 
\citet{AC_Schneier}, aunque tambi{\'{e}}n es posible utilizar 
cualquier tipo de archivo digital \citet{MP3SteganoReview}, como audio, video, etc{\'{e}}tera.

\subsection{Esteganograf{\'{i}}a en im{\'{a}}genes digitales}

Hola

An{\'{e}}cdota de Los Simpson

¿Poner la an{\'{e}}cdota antes de la definici{\'{o}}n? ¿En cursiva? ¿Tiene cabida en antecedentes o en otro lugar?

En una de las escenas del episodio "My mother the Carjacker" [?Simpson ]de la famosa serie Los Simpson [¿OtraReferencia?], Homero le{\'{i}}a con cierto inter{\'{e}}s un art{\'{i}}culo de comida, a lo que posteriormente descubre que la primera letra de las primeras l{\'{i}}neas de dicho art{\'{i}}culo formaba su nombre (en su idioma original) y que las consiguientes l{\'{i}}neas formaban una sentencia espec{\'{i}}fica, en espec{\'{i}}fico una direcci{\'{o}}n de la ciudad de Springfield Confundido, Homero piensa que alguien trata de comunicarse con {\'{e}}l y decide investigar el asunto.

A pesar de que el tema principal de este episodio de Los Simpson no se enfocaba en esta peculiar forma de comunicarse, que de hecho nuca lo llegan a describir, inadvertidamente hizo conocida al gran p{\'{u}}blico una de las actividades m{\'{a}}s antiguas que se ha ligado a la historia de la humanidad: la esteganograf{\'{i}}a, el arte de ocultar mensajes.


La esteganograf{\'{i}}a moderna 

rollo: intro y \porhacer{estructura} y discutir "" y otras cosas relevantes.

mas parrafo

otro parrafo
